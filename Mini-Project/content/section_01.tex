\section{Understanding Game Physics}
We can define "Game Physics" as the way of simulating the laws of physics within a simulation or video game using programming logic to implement these laws. One could say that it is a derivative of a physics simulation but with an element of fun and creativity and used in the form of entertainment instead of research and development.

But one thing that needs to be noted is that Game Physics is not accurate as every developer changes the effects as per their needs. Another main reason is that the games take up a lot of CPU resources to process the calculations, thus increasing the CPU requirements and reducing the audience. This can usually be seen in competitive FPS games such as Counter-Strike and Valorant as their main focus is to reach the game to as many audiences as possible, thus cutting down or reducing the physics in the game.

One thing we need to understand is that video games have always had physics, but how it's used has changed. Initially, each game had its physics programmed individually. Now, games are more complex and have multiple physics simulations running per frame, requiring a game physics engine, a universal code adapted by developers to fit specific games. The newer game physics engine (such as Unity, Unreal Engine \& Gadot) have multiple physics simulations already coded into them.