\section{Car Racing Games}
Racing Games can be defined as a genre of video games in which a player takes part in a racing competition. This may vary from the type of racing game which are:
\begin{enumerate}
    \item \textbf{Arcade-style Racing} \\
	This genre has less emphasis on realistic handling or physics, and more emphasis on fast-paced action and speed. The physics in these games is pretty liberal as its objective is to keep the player on the track and have fun. A few examples of this are that a car will have to reduce its speed in order to make the turn but in this sub-genre, the player can just somehow make the turn without slowing down much or they can just drift through the turn. The same goes with collisions as in these games the collisions are either exaggerated or just very minimal.
	
	% Another aspect is in terms of its world, where it is usually just famous race tracks or just an open world where in can roam around the world to explore and race at different checkpoints. Some famous Arcade Racing Games are:
    % \begin{itemize}[noitemsep,topsep=-5pt]
    %     \item Forza Horizon Series
    %     \item Need for Speed Series
    %     \item Midnight Club series
    %     \item Burnout series
    % \end{itemize}

    \item \textbf{Simulation Racing} \\ 
    Simulation racing games aim to replicate the handling and mechanics of a real-life vehicle. They often license real cars and race tracks to make them more realistic. As simulating the real-world experience is key the games tend to focus on the game physics and car physics more than visuals and other things. 

	But if a player with not have much experience in this genre they can use various assists to make the experience enjoyable. Some common assists found are traction control(TC), Anti-lock brakes (ABS), steering assistance, damage resistance, clutch assistance, etc.

	Another aspect these games focus on is the "Sound", as funny as it may sound is an important aspect of the game. Players want the engine and tire sounds to be physically happening with the car i.e. it matches with the surroundings. The three main elements of car audio are intake, exhaust, and internal engine sounds. 
    % Some popular Simulation Racing Games are:
    % \begin{itemize}[noitemsep,topsep=-5pt]
    %     \item Forza Motorsport series
    %     \item Gran Turismo series
    %     \item Assetto Corsa
    %     \item rFactor 2
    % \end{itemize}

    % \item \textbf{Kart Racing} \\ 
    % These games have simplified driving mechanics with boosters/power-ups for speed, bombs, etc. The tracks have unusual designs with multiple obstacles and action elements. Some popular Kart Racing Games are:
    % \begin{itemize}[noitemsep,topsep=-5pt]
    %     \item Mario Kart
    %     \item Crashing Race
    % \end{itemize}

    % \item \textbf{Futuristic Racing} \\ 
    % This genre is pretty uncommon but focuses on players using sci-fi vehicles to race against the clock. A common element seen in these games is the vehicular combat features. Some popular Kart Racing Games are:
    % \begin{itemize}[noitemsep,topsep=-5pt]
    %     \item F-Zero
    %     \item Wipeout
    % \end{itemize}

\end{enumerate}

% Now that we have the understanding of a basic understanding of the types of racing games, we move on to the Case Study on Car Racing Games where we compare 5 games based on their experience and physics and try to see the complexity of the game and physics engine.